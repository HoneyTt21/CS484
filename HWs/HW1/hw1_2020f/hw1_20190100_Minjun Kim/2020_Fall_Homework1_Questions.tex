%%%%%%%%%%%%%%%%%%%%%%%%%%%%%%%%%%%%%%%%%%%%%%%%%%%%%%%%%%%%%%%%%%%%%
%
% CS484 Written Question Template
%
% This is a LaTeX document. LaTeX is a markup language for producing 
% documents. Your task is to fill out this document, then to compile 
% it into a PDF document. 
%
% 
% TO COMPILE:
% > pdflatex thisfile.tex
%
% If you do not have LaTeX and need a LaTeX distribution:
% - Personal laptops (all common OS): www.latex-project.org/get/
% - We recommend miktex (https://miktex.org/) for latex engine,
%   and TeXstudio(http://www.texstudio.org/) for latex editor.
%   You should install both programs for editing latex.
%   Or you can use Overleaf (https://www.overleaf.com/) which is 
%   an online latex editor.
%
% If you need help with LaTeX, please come to office hours. 
% Or, there is plenty of help online:
% https://en.wikibooks.org/wiki/LaTeX
%
% Good luck!
% Min and the CS484 staff
%
%%%%%%%%%%%%%%%%%%%%%%%%%%%%%%%%%%%%%%%%%%%%%%%%%%%%%%%%%%%%%%%%%%%%%

\documentclass[11pt]{article}

\usepackage[english]{babel}
\usepackage[utf8]{inputenc}
\usepackage[colorlinks = true,
            linkcolor = blue,
            urlcolor  = blue]{hyperref}
\usepackage[a4paper,margin=1.5in]{geometry}
\usepackage{stackengine,graphicx}
\usepackage{fancyhdr}
\setlength{\headheight}{15pt}
\usepackage{microtype}
\usepackage{times}

% From https://ctan.org/pkg/matlab-prettifier
\usepackage[numbered,framed]{matlab-prettifier}

\frenchspacing
\setlength{\parindent}{0cm} % Default is 15pt.
\setlength{\parskip}{0.3cm plus1mm minus1mm}

\pagestyle{fancy}
\fancyhf{}
\lhead{Homework 1 Questions}
\rhead{CS484}
\rfoot{\thepage}

\date{}

\title{\vspace{-1cm}Homework 1 Questions}


\begin{document}
\maketitle
\vspace{-2cm}
\thispagestyle{fancy}

\section*{Instructions}
\begin{itemize}
  \item Compile and read through the included MATLAB tutorial.
  \item 2 questions.
  \item Include code.
  \item Feel free to include images or equations.
  \item Please make this document anonymous.
  \item \textbf{Please use only the space provided and keep the page breaks.} Please do not make new pages, nor remove pages. The document is a template to help grading.
  \item If you really need extra space, please use new pages at the end of the document and refer us to it in your answers.
\end{itemize}


\section*{Submission}
\begin{itemize}
	\item Please zip your folder with \textbf{hw1\_student id\_name.zip} $($ex: hw1\_20201234\_Peter.zip$)$
	\item Submit your homework to \href{http://klms.kaist.ac.kr/course/view.php?id=109597}{KLMS}.
	\item An assignment after its original due date will be degraded from the marked credit per day: e.g., A will be downgraded to B for one-day delayed submission.
\end{itemize}

\pagebreak


\section*{Questions}


%%%%%%%%%%%%%%%%%%%%%%%%%%%%%%%%%%%

% Please leave the pagebreak
\paragraph{Q1:} We wish to set all pixels that have a brightness of 10 or less to 0, to remove sensor noise. However, our code is slow when run on a database with 1000 grayscale images.

\emph{Image:} \href{grizzlypeakg.png}{grizzlypeakg.png}

\begin{lstlisting}[style=Matlab-editor]
A = imread('grizzlypeakg.png');
[m1,n1] = size( A );
for i=1:m1
    for j=1:n1
        if A(i,j) <= 10
            A(i,j) = 0;
        end
    end
end
\end{lstlisting}

\paragraph{Q1.1:} How could we speed it up?

%%%%%%%%%%%%%%%%%%%%%%%%%%%%%%%%%%%
\paragraph{A1.1:} Your answer here.

 By using both vectorization and logical indexing, we can speed the code more efficiently. The part of A which its brightness is 10 or less could be specified as an logical array, let us call it B. Logical array such as B functions more efficiently than using for loops. Plus, using B as a factor of the function varying the brightness, it would be more efficient using vectorization effect. 



%%%%%%%%%%%%%%%%%%%%%%%%%%%%%%%%%%%

% Please leave the pagebreak
\pagebreak
\paragraph{Q1.2:} What factor speedup would we receive over 1000 images? Please measure it.

Ignore file loading; assume all images are equal resolution; don't assume that the time taken for one image $\times1000$ will equal $1000$ image computations, as single short tasks on multitasking computers often take variable time.

%%%%%%%%%%%%%%%%%%%%%%%%%%%%%%%%%%%
\paragraph{A1.2:} Your answer here.

Before speeding up, using the code written below, the result of elapsed time was calculated as 34.8544587s as an average of running the program 10 times.

\begin{lstlisting}[style=Matlab-editor]
tic;
for p=1:1000

    A = imread("grizzlypeakg.png");
    [m1, n1] = size(A);

    for i=1:m1
        for j=1:n1
            if A(i,j) <= 10
                A(i,j) = 0;
            end
        end
    end
end
toc;
\end{lstlisting}

To speed up, I've written the code below. The elapsed time is 16.4487209s as an average of running the program 10 times.

\begin{lstlisting}[style=Matlab-editor]
tic;
    for p=1:1000
        A = imread("grizzlypeakg.png");
        [m1, n1] = size(A);

        B = A <= 10;
        A(B)=0;
    end
toc;
\end{lstlisting}

%%%%%%%%%%%%%%%%%%%%%%%%%%%%%%%%%%%

% Please leave the pagebreak
\pagebreak
\paragraph{Q1.3:} How might a speeded-up version change for color images? Please measure it.

\emph{Image:} \href{grizzlypeak.jpg}{grizzlypeak.jpg}

%%%%%%%%%%%%%%%%%%%%%%%%%%%%%%%%%%%
\paragraph{A1.3:} Your answer here.

Before speeding up, using the code written below, the result of elapsed time was calculated as 7.6954872s as an average of running the program 10 times.

\begin{lstlisting}[style=Matlab-editor]
tic;
A = imread("grizzlypeak.jpg");
[m1, n1] = size(A);
for i=1:m1
    for j=1:n1
        if A(i,j) <= 10
            A(i,j) = 0;
        end
    end
end 
toc;
\end{lstlisting}

To speed up, I've written the code below. The elapsed time is 0.0734264s as an average of running the program 10 times.

\begin{lstlisting}[style=Matlab-editor]
tic;
    A = imread("grizzlypeak.jpg");
    [m1, n1] = size(A);

    B = A <= 10;
    A(B)=0;
toc;
\end{lstlisting}

%%%%%%%%%%%%%%%%%%%%%%%%%%%%%%%%%%%

% Please leave the pagebreak
\pagebreak
\paragraph{Q2:} We wish to reduce the brightness of an image but, when trying to visualize the result, all we sees is white with some weird ``corruption'' of color patches.

\emph{Image:} \href{gigi.jpg}{gigi.jpg}

\begin{lstlisting}[style=Matlab-editor]
I = double( imread('gigi.jpg') );
I = I - 20;
imshow( I );
\end{lstlisting}

\paragraph{Q2.1:} What is incorrect with this approach? How can it be fixed while maintaining the same amount of brightness reduction?

%%%%%%%%%%%%%%%%%%%%%%%%%%%%%%%%%%%
\paragraph{A2.1:} Your answer here.

imshow function requires single or double values between 0~1 to show proper image. But using single or double function, it converts the image into a range of 0-255. It could be fixed by using im2single or im2double functions instead of single or double functions, which im2double would suit this case.

%%%%%%%%%%%%%%%%%%%%%%%%%%%%%%%%%%%

% Please leave the pagebreak
\pagebreak
\paragraph{Q2.2:} Where did the original corruption come from? Which specific values in the original image did it represent?

%%%%%%%%%%%%%%%%%%%%%%%%%%%%%%%%%%%
\paragraph{A2.2:} Your answer here.

Since when we convert image using double function, it converts each RGB pixels into 0-255 range, what we can see as the original corruption is the pixels of original image with its brightness in a range of 20-21 since imshow function can display double values between 0 and 1.

So, in specific, it would be values between 20-21 if the original image is converted as a 0-255 format.

%%%%%%%%%%%%%%%%%%%%%%%%%%%%%%%%%%%

\end{document}
