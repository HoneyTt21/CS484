%%%%%%%%%%%%%%%%%%%%%%%%%%%%%%%%%%%%%%%%%%%%%%%%%%%%%%%%%%%%%%%%%%%%%%%%%%%%%%%%%%%%%%%%%%%%%%%%
%
% CS484 Written Question Template
%
% Acknowledgements:
% The original code is written by Prof. James Tompkin (james_tompkin@brown.edu).
% The second version is revised by Prof. Min H. Kim (minhkim@kaist.ac.kr).
%
% This is a LaTeX document. LaTeX is a markup language for producing 
% documents. Your task is to fill out this document, then to compile 
% it into a PDF document. 
%
% 
% TO COMPILE:
% > pdflatex thisfile.tex
%
% If you do not have LaTeX and need a LaTeX distribution:
% - Personal laptops (all common OS): www.latex-project.org/get/
% - We recommend latex compiler miktex (https://miktex.org/) for windows,
%   macTex (http://www.tug.org/mactex/) for macOS users.
%   And TeXstudio(http://www.texstudio.org/) for latex editor.
%   You should install both compiler and editor for editing latex.
%   The another option is Overleaf (https://www.overleaf.com/) which is 
%   an online latex editor.
%
% If you need help with LaTeX, please come to office hours. 
% Or, there is plenty of help online:
% https://en.wikibooks.org/wiki/LaTeX
%
% Good luck!
% Min and the CS484 staff
%
%%%%%%%%%%%%%%%%%%%%%%%%%%%%%%%%%%%%%%%%%%%%%%%%%%%%%%%%%%%%%%%%%%%%%%%%%%%%%%%%%%%%%%%%%%%%%%%%
%
% How to include two graphics on the same line:
% 
% \includegraphics[width=0.49\linewidth]{yourgraphic1.png}
% \includegraphics[width=0.49\linewidth]{yourgraphic2.png}
%
% How to include equations:
%
% \begin{equation}
% y = mx+c
% \end{equation}
% 
%%%%%%%%%%%%%%%%%%%%%%%%%%%%%%%%%%%%%%%%%%%%%%%%%%%%%%%%%%%%%%%%%%%%%%%%%%%%%%%%%%%%%%%%%%%%%%%%

\documentclass[11pt]{article}

\usepackage[english]{babel}
\usepackage[utf8]{inputenc}
\usepackage[colorlinks = true,
            linkcolor = blue,
            urlcolor  = blue]{hyperref}
\usepackage[a4paper,margin=1.5in]{geometry}
\usepackage{stackengine,graphicx}
\usepackage{fancyhdr}
\setlength{\headheight}{15pt}
\usepackage{microtype}
\usepackage{times}
\usepackage{booktabs}

% From https://ctan.org/pkg/matlab-prettifier
\usepackage[numbered,framed]{matlab-prettifier}

\frenchspacing
\setlength{\parindent}{0cm} % Default is 15pt.
\setlength{\parskip}{0.3cm plus1mm minus1mm}

\pagestyle{fancy}
\fancyhf{}
\lhead{Homework Writeup}
\rhead{CS 484}
\rfoot{\thepage}

\date{}

\title{\vspace{-1cm}Homework 2 Writeup}


\begin{document}
\maketitle
\vspace{-3cm}
\thispagestyle{fancy}

\section*{Instructions}
\begin{itemize}
  \item Describe any interesting decisions you made to write your algorithm.
  \item Show and discuss the results of your algorithm.
  \item Feel free to include code snippets, images, and equations.
  \item Use as many pages as you need, but err on the short side If you feel you only need to write a short amount to meet the brief, th
  
  \item \textbf{Please make this document anonymous.}
\end{itemize}

\section*{In the beginning...}

To understand the concept of the question, as to implement convolution of imfilter() function, understanding convolution would be the most important part. As researched, convolution is a process of transforming the image by adding elements of given image to its neighbors which is weighted by the filter or kernel, using the form of mathematical convolution. 

In such results, I could understand that the task is to implement mathematical convolution in MATLAB terms. Mathematical convolution is given as Equation 1

\begin{equation}
f(t) * g(t) =  \int_{-\infty}^{\infty} f(x)g(t-x) \,dx \
\label{eq:one}
\end{equation}

\section*{Interesting Implementation Detail}

Two of the most interesting implementation details were about the difference between correlation and convolution, and implementing for both color and gray-scaled pictures. 

Since convolution requires rotating 180 degrees compared to correlation, I used rot90() function to implement it. 

\begin{lstlisting}[style=Matlab-editor]
filter = rot90(filter,2);
\end{lstlisting}

To implement for both gray-scale and RGB-scaled photos, I divided cases for the third factor of size(image), so that if there it is a RGB-based picture, I can handle it.

	\begin{lstlisting}[style=Matlab-editor]
if(RGB == 3) 
    
    output = zeros([imgRows imgCols 3]);
        
    for k = 1:3
        padImage = padarray(image(:,:,k),[fix(filRows/2) fix(filCols/2)]);
        for i = 1:imgRows
            for j = 1:imgCols
                Temp = padImage(i:i-1+filRows,j:j-1+filCols).*filter;
                output(i,j,k) = sum(Temp(:));
            end
        end
    end
        
else
(...)
    \end{lstlisting}



\section*{A Result}

As the result, I could successfully get the same data as given by the assignment introduction, which means I succeeded to implement two functions as given task. 

My functions could convolute images and hybrid them in final.

\end{document}
